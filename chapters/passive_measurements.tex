One important aspect of any measurement based TCP experiment, is choosing the environment in which the experiment is to be conducted. In this section, different platforms are considered. We discus pros and cons of each platform and give a high level overview of the techniques involved.  

\subsection{Testbeds}

A testbed refers to a relatively small isolated network that is fully controlled by the conductors of the experiment~\cite{Allman99}. Testbeds provide total control of the experiment unlike live internet tests, making the experiment possibly more robust and reproducible. The main downside of a testbed is the cost: creating a complex setting with a rich topology can cost a lot of money. 

Compared to simulated settings, testbeds tend to yield more accurate results, since real systems are being tested. 

\subsection{Live Internet Tests}

Live Internet tests are experiments, where a mesh of hosts on the Internet, controlled by the experimenters, communicate with one another. There are various ways to set up such an experiment, the most complete of which is to have each of the \textit{N} hosts send data to one another, giving \textit{$O(N^2)$} different network paths to measure with~\cite{Allman99}.

In this type of experiment, the experimenters have no control over the underlying network, making it impossible to measure the effect that the generated traffic has on other traffic of the network on the same path. The main benefit of live internet tests is that they are quite simple and inexpensive to implement and they can be used to run tests in a wide variety of network conditions.

\subsection{Simulation} 

Multiple scriptable general purpose simulation frameworks, such as ns2~\cite{Singh12} exist today. Some researchers also write their specialized simulators from the ground up~\cite{Allman99}. Simulators offer a major cost and time savings compared to real world measurements~\cite{Allman99}, especially so when an experiment contains multiple parameters that must be varied. Mutating the network topology, for example, can be quite tedious in a real world setting. Another advantage is that a simulation can test TCP performance in a network that does not currently exist in real world, making it possible to predict how an algorithm may perform in the future.   

One major disadvantage posed by simulations, is that there is always a certain amount of loss of detail involved. Estimating whether or not the lost details matter to the outcome of the experiment may be difficult. Thus, results of simulations are often at least partially validated in a real world experiments~\cite{Allman99}.    

\subsection{Emulation}

An emulation refers to an experiment that is a mixture of a real world experiment, usually a testbed, and a simulation~\cite{Allman99}. An emulator models a part of a network path between two real nodes. As an example, emulation may be useful for modeling part a network that cannot be easily incorporated in a physical experiment, such as a satellite channel~\cite{Allman99}. This allows the researchers to isolate the simulation to only a part of the whole experiment, making the results as credible as a real world experiment, so long as the isolated simulation is reliable enough.   